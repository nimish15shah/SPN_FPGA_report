%!TEX root = ./thesis.tex

\chapter{Introduction}
\label{cha:intro}

\ns{The title of the thesis gives an impression that error analysis is computed using FPGAs. Somehow imply that they are two independent aspects of the project.}

\Glspl{spn} are a brand new deep network architecture for \sout{graphical} \ns{probabilistic inference} inference \cite{spns}. This architecture is designed to make the partition function tractable \ns{Describe a bit what is this and how is it interesting? it comes out of blue right now.}. \ns{Motivate more about spns, application etc. before explaining your contribution}. In this work, I study the error bound of the \gls{spn} architecture and observe how it grows in each layer \ns{also mention FPGA implementation}.

\ns{discuss a bit about why low precision arithmetic is useful (low energy etc.) and drawbacks (impact on accuracy), and how this work is trying to quatify the impact on accuracy systematically}. Two representations are studied: posit \cite{posit_std} and floating point \cite{float_std}. These two number representations are used to represent \sout{floating point} \ns{real} numbers in computers. The main difference between these representations is that the exponent field is encoded using a variable length for posit. Therefore there would be cases where posit is better than floating point because exponent fields would be saved, and cases where floating point is better than posit because exponent fields would be wasted. the code to compute the error bound \ns{this should be in experiments section} is available at \url{https://github.com/gennartan/spn_sw}.

A hardware implementation of \glspl{spn} in \gls{fpga} is also provided to prove the efficiency of posit representation in a real application. The entire design is made available at \url{https://github.com/gennartan/Arithmetic_circuit_Posit_FPGA/}. \ns{again, this link should come latter, or perhaps in a footnote.}

\todo[inline]{Introduction should be longer and explain a bit more the problematic...}
\ns{Agree. Give more general details, before diving in specifics about your work. Always motivate before explaining why whatever you implemented is important}