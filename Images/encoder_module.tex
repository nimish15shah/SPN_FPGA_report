%TEX root = ./overwiew.tex

\documentclass[varwidth]{standalone}
% http://texample.net/tikz/examples/inertial-navigation-system/

\usepackage{tikz}
\usetikzlibrary{positioning}
\usetikzlibrary{shapes,arrows}

\begin{document}

\tikzstyle{module}=[draw, fill=blue!20, text width=5em, text centered, minimum height=2cm]
\tikzstyle{module_def}=[fill=yellow!20,rounded corners, draw=black!50, dashed]
\tikzstyle{signal_name} = [above, text width=10em]
\tikzstyle{module_name} = [above right, text width=10em]

\def\blockdist{2.5cm}
\def\edgedist{1.5cm}

\pgfdeclarelayer{background}
\pgfdeclarelayer{foreground}
\pgfsetlayers{background,main,foreground}

\begin{tikzpicture}

    \node (encoder) [module,minimum height=4cm] {Posit\\Encoder};
    \path (encoder.150)+(-\blockdist,0) node (exponent) [module] {Exponent\\encoder};
    \path (encoder.-130)+(-2*\blockdist,0) node (input) [] {};
    \path (encoder.0)+(0.5*\blockdist,0) node (output) [] {};

    \path [draw, ->] (exponent.30) -- node[signal_name, align=center] {regime} (encoder.west |- exponent.30);
    \path [draw, ->] (exponent.-30) -- node[signal_name, align=center] {exp} (encoder.west |- exponent.-30);
    \path [draw, ->] (input.west) -- node[signal_name, align=center] {significand} (encoder.west |- input);
    \path [draw, <-] (exponent.west) -- node[signal_name, align=center] {exponent} (input.west |- exponent.west);
    \path [draw,->] (encoder.east) -- node[signal_name, align=center] {posit} (output);

    \begin{pgfonlayer}{background}
    	\path (output.west |- encoder.north)+(0.2,0.2) node (a) {};
    	\path (input.west |- encoder.south)+(-0.2,-0.2) node (b) {};
        \path [module_def] (a) rectangle (b);
        \path (b) node[module_name] (name) {Encoder Module};
    \end{pgfonlayer}


\end{tikzpicture}

\end{document}

