%TEX root = ./overwiew.tex

\documentclass{standalone}
% http://texample.net/tikz/examples/inertial-navigation-system/

\usepackage{tikz}
\usetikzlibrary{positioning}
\usetikzlibrary{shapes,arrows}

\begin{document}

% \tikzstyle{module}=[draw, fill=blue!20, text width=5em, text centered, minimum height=2cm]
% \tikzstyle{module_def}=[fill=yellow!20,rounded corners, draw=black!50, dashed]
% \tikzstyle{signal_name} = [above, text width=10em]
% \tikzstyle{module_name} = [above right, text width=10em]

\tikzstyle{base}=[draw, circle, minimum height=1cm]

\def\blockdist{1.5cm}
\def\edgedist{1.5cm}

\begin{tikzpicture}

    \node (trueNode) [base] {+};
	\node () [above right=0cm and 0cm of trueNode] {\color{gray} Id};
    \node (mul1) [base, below left =of trueNode] {$\times$};
    \node (mul2) [base, below right =of trueNode] {$\times$};
    \node (lit1) [below left=1.5cm and 0.35cm of mul1] {$X_i$};
    \node () [above right=0cm and 0cm of lit1] {\color{gray} Lit};
    \node (w1) [below right=1.5cm and 0.35cm of mul1] {p};
    \node (lit2) [below left=1.5cm and 0.35cm of mul2] {$\bar{X_i}$};
    \node () [above right=0cm and 0cm of lit2] {\color{gray} Lit};
    \node (w2) [below right=1.5cm and 0.35cm of mul2] {1-p};


   	\path [draw, ->] (trueNode) -- (mul1);
   	\path [draw, ->] (trueNode) -- (mul2);
   	\path [draw, ->] (mul1) -- (lit1);
   	\path [draw, ->] (mul1) -- (w1);
   	\path [draw, ->] (mul2) -- (lit2);
   	\path [draw, ->] (mul2) -- (w2);

\end{tikzpicture}

\end{document}

